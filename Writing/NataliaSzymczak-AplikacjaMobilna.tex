\documentclass[12pt,twoside]{report}
\usepackage[margin=2.5cm]{geometry}
\usepackage[utf8]{inputenc}
\usepackage[T1]{fontenc}
\usepackage{polski}
\usepackage{indentfirst}
\usepackage[shortlabels]{enumitem}
\usepackage{multirow}
\usepackage{makecell}
\usepackage{float}
\begin{document}
	\begin{titlepage}
		
		\begin{center}
			\begin{figure}[t]
				\centering
			%	\includegraphics[width=4.5cm,height=4.5cm]{LogoUO.jpg}
			\end{figure}
		\end{center}
		
		\begin{center}
			{\LARGE  \bf \textsc{UNIWERSYTET OPOLSKI}}
		\end{center}
		\vspace{0.2cm}
		\begin{center}
			{\large \textsc{Wydział Matematyki, Fizyki i Informatyki}}
		\end{center}
		%\vspace{0.2cm}
		\begin{center}
			{\Large \textsc{Instytut Informatyki}}
		\end{center}
		\vspace{0.5cm}
		\begin{center}
			\large    \textsc{Praca iżynierska}
		\end{center}
		\vspace{0.4cm}
		\begin{center}
			\large \textbf{Natalia Szymczak}
		\end{center}
		
		\vspace{0.4cm}
		\begin{center}
			\Large     \textbf{Aplikacja mobilna dla posiadaczy koni}
		\end{center}
		\vspace{0.1cm}
		
		\begin{center}
			\large     \textsc{Mobile aplication for horse owner}
		\end{center}
		\vspace{1.3cm}
		
		\begin{flushright}
			{\large Praca wykonana pod kierunkiem\bigskip
				
				{\bf }} 
			dr Jacka Iwańskiego
		\end{flushright}
		\vspace{0.7cm}
		\begin{center}
			{\large OPOLE 2022}
		\end{center}
	\end{titlepage}

	\thispagestyle{empty}
	\mbox{}
	
	\begin{quote}{\small 
			\noindent
			
			\bigskip
			\noindent
			\textbf{Streszczenie:} 
			
			
			\noindent
			\newline
			\textbf{}
			\vspace{5pt}
			
			\noindent
			\newline
			\textbf{Abstract:} 
			\vspace{5pt}
			
			\vspace{5pt}
			\noindent
			\newline
			\textbf{Keywords:} 
			\vspace{5pt}
			\bigskip
			
			\noindent 
			\textbf{Klasyfikacja tematyczna wg  MSC 2020:}}
	\end{quote}

	\mbox{}
	
	\pagestyle{plain}
	\tableofcontents
	\thispagestyle{empty}
	
	
	\newpage
	\setcounter{page}{1}
	\newpage
\chapter{Wstęp}
\chapter{Przegląd istniejących rozwiązań}
\chapter{Technologie użyte w pracy}
\section{Microsoft SQL Server 2019}
\section{Microsoft SQL Server Managemnet Studio}
\section{Microsoft Visual Studio 2022}
\section{Windows Presentation Foundation}
\section{Xamarin}
\section{Android Device Menager}
\section{NuGet}
\section{Entity Framework}
\section{Ten do wykresów}
\section{MVVM Toolkit}
\section{Xamarin Community Toolkit}
\section{ZXing.Net.Mobile.Forms}
\section{Structured Query Language}
\chapter{Specyfikacja wymagań}
\section{Opis wycinka rzeczywistości}
Aplikacja przeznaczona jest dla klubów jeździeckich, czyli organizacji zrzeszających jeźdźców startujących w danej dziedzinie sportów konnych. Aplikacja nadaje się dla klubów, których zawodnicy startują w takich dziedzinach jak:
\begin{itemize}
	\item Skoki przez przeszkody,
	\item WKKW (skrót od "Wszechstronny konkurs konia wierzchowego"),
	\item Ujeżdżenie.
\end{itemize}

W celu jak najlepszego określenia wymagań funkcjonalnych, przed napisaniem aplikacji przeprowadzono rozmowy z kilkoma osobami zaangażowanymi w to środowisko: pracownikami stadnin państwowych, właścicielami klubów, jak także z osobami prywatnie trzymającymi konie w stadninach. Po przeprowadzonych rozmowach zdecydowano się na dwie wersje aplikacji: desctopową oraz mobilną, które będą różnić się funkcjonalnościami.

Aplikacja ma na celu pomóc w gromadzeniu informacji o jeźdźcach przynależących do klubu oraz ich koniach. W aplikacji gromadzone są informacje o codziennych aktywnościach koni, ich chorobach, żywieniu oraz zawodach w których biorą udział. Naturalnie chcemy także zapisywać wyniki z tych zawodów, aby móc określić czy dany trening jest skuteczny. Z aplikacji będą korzystać zawodnicy, trenerzy, jaki i zarząd klubu.

Aby skutecznie zbierać informacje o treningach i innych aktywnościach niezbędna jest aplikacja mobilna, ponieważ dane te muszą być wprowadzane na bieżąco. Informacje o wizytach różnorakich lekarz oraz kowala także muszą być zapisywane na bieżąco podczas danej wizyty. Dlatego funkcjonalności te dotyczą jedynie aplikacji mobilnej. W aplikacji mobilnej można również sprawdzić aktualny plan żywienia swojego konia. Do tej aplikacji będą mieć dostęp jedynie osoby posiadające konie. 

W aplikacji desktopowej wyświetlane są statystyki aktywności koni danego użytkownika jak i szczegóły wizyt lekarzy i kowali. W tej aplikacji można zaplanować wyjazdy na zawody jak także szczegółowe plany żywienia swoich podopiecznych. W tej aplikacji tworzone  będą także konta użytkowników, oraz ich koni. Dostęp do funkcji tworzenia kont będzie ograniczony i posiadać go będzie jedynie administrator aplikacji.

Każdy członek klubu będzie miał swoje konto z możliwością logowania zarówno do aplikacji mobilnej jak i desktopowej. Trenerzy, właściciele klubu i inne osoby związane z klubem będą miały dostęp jedynie do aplikacji desktopowej. 


\section{Wymagania funkcjonalne}
Funkcjonalności aplikacji mobilnej oraz desktopowej nie są takie same mimo iż są podłączone do jednej bazy, więc czerpią z tego samego źródła informacji. Pomimo znaczących różnic niektóre funkcjonalności pokrywają się w obu tych produktach. 
Wymagania funkcjonalne, które muszą spełniać obie aplikacje przedstawia tabelka \ref{funkcjonalneObuApek}.
\renewcommand{\arraystretch}{1.5}
\begin{table}[h!]
	\centering
\begin{tabular}{|p{4.5cm}|p{4cm}|p{7cm}|}			
	\hline
	Wymaganie & Aktor & Opis wymagania\\
	\hline
	Logowanie do aplikacji& Trener, Członek klubu, Zarząd klubu & System pozwala na zalogowanie się po podaniu poprawnego loginu oraz hasła.\\
	\hline
	Resetowanie hasła przez email & Trener, Członek klubu, Zarząd klubu& System umożliwia resetowanie hasła przez adres e-mail. \\
	\hline
	
\end{tabular}
	\caption{Wymagania funkcjonalne obu aplikacji}
	\label{funkcjonalneObuApek}
\end{table}

Aplikacja mobilna będzie służyć użytkownikom głównie do zapisu aktualnych wydarzeń z życia stajni. Jej głównym celem jest szybkie zapisanie informacji o aktywnościach koni i ich wizytach u lekarzy, bądź kowali. Można w niej także szybko sprawdzić przygotowany plan żywienia, oraz daty zbliżających się zawodów.
Wymagania funkcjonalne dla aplikacji mobilnej zawierają poniższe tabele \ref{funkcjonalneMobilki1} i \ref{funkcjonalneMobilki2}.
\renewcommand{\arraystretch}{1.8}
\begin{table}[h!]
	\centering
	\begin{tabular}{|p{3cm}|p{3cm}|p{3cm}|p{6cm}|}			
		\hline
	   \multicolumn{2}{|l|}{Wymaganie} & Aktor & Opis wymagania\\
		\hline
		Zarządzanie aktywnościami & Dodawanie aktywności & Członek klubu &  System umożliwia zapis danych wprowadzonych przez zalogowanego użytkownika do bazy danych.\\	
		\hline	
		Zarządzanie aktywnościami & Edytowanie aktywności & Członek klubu & System umożliwia edytowanie dodanych wcześniej danych o aktywnościach. \\	
		\hline	
		Zarządzanie aktywnościami& Usuwanie aktywności & Członek klubu & System pozwala na usuwanie dodanych wcześniej aktywności. \\
		\hline
		Zarządzanie aktywnościami& Wyświetlanie aktywności & Trener, Członek klubu, Zarząd klubu& System umożliwia na przeglądanie wszystkich danych o aktywnościach danego konia zgromadzonych w bazie danych.\\
		\hline
		Zarządzanie wizytami & Dodawanie wizyt & Członek klubu& System pozwala na zapisanie danych z wizyty konia u lekarza/kowala do bazy danych.\\
		\hline
		Zarządzanie wizytami & Edytowanie wizyt & Członek klubu& System powinien umożliwić zapis zaktualizowanych danych o wizycie do bazy.\\
		\hline
		Zarządzanie wizytami & Usuwanie wizyt & Członek klubu & System powinien umożliwiać usuwanie danych o dodanych wcześniej wizytach.\\
		\hline 
		Zarządzanie wizytami & Wyświetlanie wizyt & Trener, Członek klubu, Zarząd klubu& System powinien umożliwić przeglądanie danych o wizytach zgromadzonych w bazie.\\
		\hline
		Zarządzanie wizytami & Planowanie wizyt & Członek klubu & System powinien pozawalać użytkownikom na dodanie do bazy danych o następnej wizycie, czyli umożliwić zapis wizyt jedynie z datą i opisem.\\
		\hline
		Zarządzanie wizytami & Zapisywanie zdjęcia z wizyty & Członek klubu&System powinien pozwalać na zapisywanie zdjęci z wizyt.\\
		\hline
	\end{tabular}
	\caption{Wymagania funkcjonalne aplikacji mobilnej}
	\label{funkcjonalneMobilki1}
\end{table}

\renewcommand{\arraystretch}{1.8}
\begin{table}[h!]
	\centering
	\begin{tabular}{|p{3cm}|p{3cm}|p{4cm}|p{6cm}|}			
		\hline
		\multicolumn{2}{|l|}{Wymaganie} & Aktor & Opis wymagania\\
		\hline
		Zarządzanie wizytami & Przypomnienia o wizytach& Członek klubu & System powinien wysłać powiadomienie o zbliżającej się wizycie\\
		\hline
		Zarządzanie żywieniem & Przeglądanie planów żywienia & Członek klubu & System powinien umożliwiać przeglądanie planów żywienia umieszczonych w bazie.\\
		\hline
		Zarządzanie żywieniem & Wybór planu żywienia & Członek klubu & System powinien umożliwiać wybór jednego z planów żywienia umieszczonych w bazie jako tego aktualnie używanego.\\
		\hline
		Zarządzanie zawodami & Wyświetlanie najbliższych zawodów & Członek klubu & System powinien umożliwić wyświetlanie dat najbliższych zawodów umieszczonych w bazie.\\
		\hline
		Zarządzanie zawodami & Potwierdzenie udziału w zawodach & Członek klubu & System powinien umożliwić użytkownikowi potwierdzenie swojego udziału w zawodach.\\
		\hline
		\multicolumn{2}{|l|}{Udostępnianie koni}&Członek klubu& System powinien umożliwić udostępnianie koni między użytkownikami.\\
		\hline
	\end{tabular}
	\caption{Wymagania funkcjonalne aplikacji mobilnej}
	\label{funkcjonalneMobilki2}
\end{table}
\newpage
$\ $
\newpage
Aplikacja desktop-owa przeznaczona jest zarówno dla użytkowników posiadających swoje konie jak i dla osób zarządzających klubem jeździeckim. W aplikacji desktop-owej posiadacze koni będą mogli obejrzeć zgromadzone informacje w przystępniejszej formie na dużym ekranie, stworzyć plan żywienia swojego konia, jak także przeanalizować statystki swoich koni. Osoby zarządzające klubem będą miały możliwość dodawania nowych użytkowników i koni jak także sprawdzania statystyk wszystkich koni klubowych. Szczególowe wymagania funkcjonalne dla aplikacji desktopowej zostały przedstawione w tabelach \ref{funkcjonalneDesktop1} oraz \ref{funkcjonalneDesktop2}.
\renewcommand{\arraystretch}{1.8}
\begin{table}[h!]
	\centering
	\begin{tabular}{|p{3cm}|p{3cm}|p{4cm}|p{6cm}|}			
		\hline
		\multicolumn{2}{|l|}{Wymaganie} & Aktor & Opis wymagania\\
		\hline
		Zarządzanie planami żywienia & Tworzenie planów żywienia  & Członek klubu & System umożliwia użytkownikowi stworzenie planu żywienia i zapisanie go do bazy.\\		
		\hline
		Zarządzanie planami żywienia & Edytowanie planów żywienia  & Członek klubu & System pozwala aktualizować stworzone wcześniej plany żywienia.\\
		\hline
		Zarządzanie planami żywienia & Usuwanie planów żywienia  & Członek klubu & System umożliwia usuwanie danych o stworzonych wcześniej planach żywienia.\\
		\hline
		Zarządzanie końmi& Dodawanie koni & Zarząd klubu & System umożliwia wprowadzenie danych o koniach i dodanie ich do konkretnego użytkownika\\ 		
		\hline
		Zarządzanie końmi& Usuwanie koni & Zarząd klubu & System umożliwia usuwanie koni\\ 		
		\hline
		Zarządzanie końmi& Edytowanie koni & Zarząd klubu & System umożliwia edycję danych o koniach zgromadzonych już w bazie.\\ 
		\hline
		Zarządzanie użytkownikami & Dodawanie użytkowników & Zarząd klubu & System umożliwia dodawanie danych o użytkownikach i tworzenie ich kont.\\		
		\hline
		Zarządzanie użytkownikami & Edytowanie użytkowników & Zarząd klubu & System umożliwia edytowanie danych użytkownika\\		
		\hline
		Zarządzanie użytkownikami & Usuwanie użytkowników & Zarząd klubu & System umożliwia usuwanie użytkowników\\
		\hline
		Zarządzanie użytkownikami & Zmiana hasła & Zarząd klubu,Członek klubu, Trener& System umożliwia zmianę hasła przez użytkownika.\\
		\hline
		
	\end{tabular}
\caption{Wymagania funkcjonalne aplikacji desktopowej}
\label{funkcjonalneDesktop1}
\end{table}
\renewcommand{\arraystretch}{1.8}
\begin{table}[h!]
	\centering
	\begin{tabular}{|p{3cm}|p{3cm}|p{4cm}|p{6cm}|}			
		\hline
		\multicolumn{2}{|l|}{Wymaganie} & Aktor & Opis wymagania\\
		\hline
		Zarządzanie zawodami & Dodawanie zawodów & Zarząd klubu& System pozwala na tworzenie zawodów, oraz zapraszanie do udziału w nich poszczególnych członków klubu\\	
		\hline	
		Zarządzanie zawodami & Edytowanie zawodów & Zarząd klubu& System pozwala na edycję danych o dodanych wcześniej zawodach\\
		\hline
		Zarządzanie zawodami & Usuwanie zawodów & Zarząd klubu& System pozwala na usuwanie danych o dodanych wcześniej zawodach.\\
		\hline
		\multicolumn{2}{|l|}{Przeglądanie histori wizyt}& Członek klubu, Trener, Zarząd klubu&\\
		\hline
		\multicolumn{2}{|l|}{Przeglądanie statystyk}&Członek klubu, Trener, Zarząd klubu&\\
		\hline
	\end{tabular}
	\caption{Wymagania funkcjonalne aplikacji desktopowej}
	\label{funkcjonalneDesktop2}
\end{table}
\newpage
$\ $
\newpage
\subsubsection{Przypadki użycia}
Wszystkie wymagania funkcjonalne zgromadzone w powyższych tabelach, możemy przedstawić na diagramie przypadków użycia.
\\
UML-
\\
Przypadek użycia - jest to 
\\
Diagram przypadków użycia - 
\\
\section{Wymagania niefunkcjonalne}
Wymagania niefunkcjonalne:
\newpage
\chapter{Baza danych}
\section{Model konceptualny}
Proces tworzenia bazy danych zaczynamy od modelu konceptualnego. W pierwszej fazie tworzenia go ważne jest określenie słownika pojęć, które będą następnie używane w projekcie bazy danych.
\subsubsection{Słownik pojęć}
\begin{itemize}
	\item \textbf{Użytkownik} - wszyscy członkowie klubu, trenerzy oraz zarząd klubu.
	\item \textbf{Koń} - koń należący do któregoś z członków klubu jeździeckiego, lub dzierżawiony przez niego.
	\item \textbf{Atywności} - są to czynności wykonywane przez konia w ciągu dnia, należą do nich jazdy, skoki przez przeszkody, kross, ujeżdżenie, lonża, wyjazd w teren, karuzela, padok, wyjazd na zawody, spacer, skoki luzem, padok.
	\item \textbf{Wizyty} - to wizyty wszelkich lekarzy, jak także wizyty kowali.
	\item \textbf{Udostępnianie konia} - jest to przekazanie możliwości wprowadzania danych o danym koniu przez jego właściciela innemu członkowi klubu.
\end{itemize}
Po określeniu definicji poszczególnych pojęć używanych w projekcie możemy przystąpić do tworzenia kategorii.
\subsubsection{Kategorie}
Po przeanalizowaniu wycinku rzeczywistości możemy określić jakie dane chcemy zbierać i zapisywać do bazy danych. Dane te możemy podzielić na kategorie i opisać językiem naturalnym ich cechy charakterystyczne.
\begin{enumerate}[start=1,label={\bfseries KAT:\arabic*}]
	\item 
\end{enumerate}
Po określeniu kategorii możemy określić reguły funkcjonowania naszej aplikacja.
\subsubsection{Reguły funkcjonowania}
Reguły funkcjonowania określają zasady, procedury i wytyczne jakie musi spełniać projektowana aplikacja. 
\begin{enumerate}[start=1,label={\bfseries REG\textbackslash 00\arabic*}]
	\item Konta użytkowników tworzy jedynie użytkownik "administrator".
	\item Każdy użytkownik ma określony swój typ.
	\item Każdy użytkownik może zmienić swoje hasło.
	\item O każdym użytkowniku, jak także o lekarzu i kowalu zbieramy podstawowe dane personalne.
	\item Tylko użytkownik "administrator" dodaje konie do kont użytkowników.
	\item Każdy koń ma przypisaną płeć.
	\item Każdy koń ma przypisany status.
	\item Jeden użytkownik może posiadać wiele koni.
	\item Aktywności konia może dodać jego właściciel lub osoba której właściciel udostępni konia.
\end{enumerate}
\begin{enumerate}[start=10,label={\bfseries REG\textbackslash 0\arabic*}]
	\item Koń może mieć wiele aktywności każdego dnia.
	\item Wizyty konia może dodawać tylko jego właściciel.
	\item Na wizycie jest jeden koń i jedne lekarz/kowal.
	\item Każdy lekarz ma określoną specjalizacje.
	\item Plan żywienia konia może ustalać tylko właściciel. 
	\item Koń może posiadać wiele planów żywienia, ale aktualnie może jeść tylko jeden.
	\item Plan żywienia zawiera wiele żywień.
	\item Żywienie dotyczy konkretnego typu jedzenia, podawanego o konkretnej porze (rano, południe, wieczór), który swoją jednostkę miary.
	\item Użytkownicy, którym ktoś udostępnił konia mogą tylko wyświetlić plan żywienia.
	\item Statystyki mają być tworzone na podstawie aktywności.
	\item Użytkownik "członek klubu" może przeglądać statystyki tylko swoich koni.
	\item Użytkownik "trener" lub "administrator" może przeglądać statystyki wszystkich koni.
	\item Użytkownik "trener" lub "administrator" może dodawać wyjazd na zawody dla całego klubu i zapraszać poszczególnych użytkowników.
	\item Użytkownik "członek klubu" może dodawać swoje wyjazdy na zawody.
\end{enumerate}

\subsubsection{Ograniczenia dziedzinowe}
Ograniczenia dziedzinowe to ograniczenia, które nakładane są na atrybuty w powyższych kategoriach. Wynikają one z analizy wycinka rzeczywistości i należy je uwzględnić podczas projektowania bazy danych oraz implementacji systemu.
\begin{enumerate}[start=1,label={\bfseries OGR\textbackslash 00\arabic*}]
	\item Paszport konia składa się ze znaków i cyfr postaci xxx-aaa-bb-ccccc-dd, gdzie
	\begin{itemize} 
		\item xxx - określa kraj pochodzenia konia,  
		\item aaa - oznacza kod hodowli konia, 
		\item bb- oznacza rok urodzenia konia, 
		\item ccccc - to numer paszportu konia, 
		\item dd - to numer identyfikacyjny konia w ramach hodowli.
		\end{itemize}
	\item Data wizyty konia jest wcześniejsza niż data jego urodzenia.
	\item 
\end{enumerate}
\subsubsection{Transakcje}
Transakcje są to operacje, które możemy wykonywać na danych. Mają one cztery własności, które w skrócie nazywamy ACID (ang. Atomicity, Consistency, Isolation, Durability). Transakcje mają więc następujące własności:
\begin{itemize}
	\item atomowość, inaczej niepodzielność oznacza, że transakcje muszą być wykonywane na bazie w całości. Jeśli transakcja nie zostanie poprawnie przeprowadzona należny przywrócić stan bazy z przed jej wykonania.
	\item spójność, po wykonaniu transakcji baza powinna być nadal spójna.
	\item izloacja, oznacza że tranzakcje nie mogą być od siebie zależne.
	\item trwałość, oznacza że dane po transakcji zostają zapisane w bazie i są zachowane na stałe.
\end{itemize}	
Transakcje występujące w aplikacji:
\begin{enumerate}[start=1,label={\bfseries TRA\textbackslash00\arabic*}]
	
	\item \textbf{Dodanie aktywności }\\
	\textit{Opis}: Zadaniem transakcji jest dodanie danych o aktywności konia. Aktywności konia może dodać jedynie członek klubu, który jest jego właścicielem lub osoba której został on udostępniony.\\
	\textit{Uwarunkowania}: Aktywność musi zawierać dane o tym kto ją wprowadził, jakiego konia ona dotyczy, w jakim dniu została wykonana, oraz czas jej trwania.\\
	\textit{Wejście}:
		\begin{itemize}
			\item U - Dane nowej aktywności
			\item BD - Dane aktywności
		\end{itemize} 
	\textit{Wyjście}:
		\begin{itemize}
			\item U - Komunikat
			\item BD - Dane aktywności
		\end{itemize} 
	
	\item \textbf{Edycja aktywności }\\
	\textit{Opis}: \\
	\textit{Uwarunkowania}: \\
	\textit{Wejście}:
	\begin{itemize}
		\item U - 
		\item BD -
	\end{itemize} 
	\textit{Wyjście}:
	\begin{itemize}
		\item U - 
		\item BD -
	\end{itemize} 
\end{enumerate}
\newpage
\renewcommand{\arraystretch}{1.5}
\subsubsection{Encje}
Po określeniu kategorii, reguł funkcjonowania, ograniczeń dziedzinowych i transakcji należy przystąpić do tworzenia encji i relacji między nimi.
Definicje??
\\
\begin{enumerate}[start=1,label={\bfseries ENC\textbackslash00\arabic*}]
	
	\item \textbf{Activity}
	
	\textit{Semantyka encji} - Encja zawierająca aktywności konia.
	
	\begin{table}[h!]
		\centering
		\begin{tabular}{|l|l|l|c|}
			\hline
			Nazwa atrybutu & Opis atrybutu & Typ & OBL(+) \\
			& & &  OPC(-) \\
			\hline
			activityID & Numer identyfikujący aktywności & Liczba naturalna & + \\
			\hline
			date & Data aktywności & Data & + \\
			\hline
			description & Opis aktywności & typ znakowy & - \\
			\hline
			time &  Czas trwania aktywności & Czas & + \\
			\hline
			intensivity & Intensywność aktywności & Liczba naturalna & + \\
			\hline
			satisfaction & Satysfakcja aktywności & Liczba naturalna & + \\
			\hline
			activityType &  Typ aktywności & Liczba naturalna & + \\
			\hline
		\end{tabular}
		\caption{Wykaz atrybutów encji typu Activity }
	\end{table}
	Klucze kandydujące: activityID \\
	Klucz główny: activityID \\
	Charakter encji: encja silna \\
	

	
	
	\item \textbf{Competition}\\ \\
	\textit{Semantyka encji} - encja zawierająca dane o zawodach.
	
	\begin{table}[h!]
		\centering
		\begin{tabular}{|l|l|l|c|}
			\hline
			Nazwa atrybutu & Opis atrybutu & Typ & OBL(+) \\
			& & &  OPC(-) \\
			\hline
			competitionID & Numer identyfikujący zawody & Liczba naturalna & + \\
			\hline
			spot & Miejsce wizyty & max. znaków 50 & - \\
			\hline
			description & Opis zawodów & Typ znakowy & - \\
			\hline
			rank & Ranga zawodów &  max. znaków 50 & - \\
			\hline
		\end{tabular}
		\caption{Wykaz atrybutów encji typu Competition }
	\end{table}
	Klucze kandydujące: competitionID \\
	Klucz główny: competitionID \\
	Charakter encji: encja silna \\
	
	
	\item \textbf{CustomNotification}
	
	\textit{Semantyka encji} - Encja zawierająca powiadomienia.
	
	\begin{table}[h!]
		\centering
		\begin{tabular}{|l|l|l|c|}
			\hline
			Nazwa atrybutu & Opis atrybutu & Typ & OBL(+) \\
			& & &  OPC(-) \\
			\hline
			notificationID & Numer identyfikujący powiadomienie& Liczba naturalna & + \\
			\hline
			title &  Tytuł powiadomienia & max. znaków 50 & + \\
			\hline
			description &  Data kończąca udostępnienie & typ znakowy & + \\
			\hline
			sendDate &  Data wysłania & Data & + \\
			\hline
			createdDate &  Data stworzenia & Data & + \\
			\hline
		\end{tabular}
		\caption{Wykaz atrybutów encji typu CustomNotification }
	\end{table}
	Klucze kandydujące: notificationID \\
	Klucz główny: notificationID \\
	Charakter encji: encja silna \\
	
	
	\item \textbf{Doctor}\\ \\
	\textit{Semantyka encji} - encja opisująca lekarzy łącząca ich dane kontaktowe ze specjalizacją.
	\begin{table}[h!]
		\centering
		\begin{tabular}{|l|l|l|c|}
			\hline
			Nazwa atrybutu & Opis atrybutu & Typ & OBL(+) \\
			& & &  OPC(-) \\
			\hline
			doctorID & Numer identyfikujący doktora & Liczba naturalna & + \\
			\hline
		\end{tabular}
		\caption{Wykaz atrybutów encji typu Doctor }
	\end{table}\\
	Klucze kandydujące: doctorID \\
	Klucz główny: doctorID \\
	Charakter encji: encja silna \\
	
	\item \textbf{DoctorSpecialisation}\\ \\
	\textit{Semantyka encji} - encja słownikowa zawiera nazwy specjalizacji lekarzy jak także kowali.\\
	\begin{table}[h!]
		\centering
		\begin{tabular}{|l|l|l|c|}
			\hline
			Nazwa atrybutu & Opis atrybutu & Typ & OBL(+) \\
			& & &  OPC(-) \\
			\hline
			specialisationID & Numer identyfikujący specjalizacje & Liczba naturalna & + \\
			\hline
			name & Nazwa specjalizacji & max. znaków 50 & + \\
			\hline
		\end{tabular}
		\caption{Wykaz atrybutów encji typu DoctorSpecialization }
	\end{table}
	Klucze kandydujące: specializationID \\
	Klucz główny: specializationID \\
	Charakter encji: encja silna \\
	
	\item \textbf{Eat}
	
		\textit{Semantyka encji} - Encja zawierająca informacje o planach żywienia i ich przynależności do koni.
	
	\begin{table}[h!]
		\centering
		\begin{tabular}{|l|l|l|c|}
			\hline
			Nazwa atrybutu & Opis atrybutu & Typ & OBL(+) \\
			& & &  OPC(-) \\
			\hline
			eatID & Numer identyfikujący jedzenie & Liczba naturalna & + \\
			\hline
			isActive &  Czy obecny plan jest w użyciu? &  & + \\
			\hline
		\end{tabular}
		\caption{Wykaz atrybutów encji typu Feed }
	\end{table}
	Klucze kandydujące: eatID \\
	Klucz główny: eatID \\
	Charakter encji: encja silna \\
	
	
	\item \textbf{Feed}
	
	\textit{Semantyka encji} - Encja zawierająca informacje o jedzeniu koni.
	
	\begin{table}[h!]
		\centering
		\begin{tabular}{|l|l|l|c|}
			\hline
			Nazwa atrybutu & Opis atrybutu & Typ & OBL(+) \\
			& & &  OPC(-) \\
			\hline
			feedID & Numer identyfikujący jedzenie & Liczba naturalna & + \\
			\hline
			amount &  Ilość jedzenia w porcji & Liczba zmiennoprzecinkowa & + \\
			\hline
		\end{tabular}
		\caption{Wykaz atrybutów encji typu Feed }
	\end{table}
	Klucze kandydujące: feedID \\
	Klucz główny: feedID \\
	Charakter encji: encja silna \\
	
	\item \textbf{Forage}
	
	\textit{Semantyka encji} - Encja zawierająca informacje o paszy dla koni.
	
	\begin{table}[h!]
		\centering
		\begin{tabular}{|l|l|l|c|}
			\hline
			Nazwa atrybutu & Opis atrybutu & Typ & OBL(+) \\
			& & &  OPC(-) \\
			\hline
			forageID & Numer identyfikujący paszy & Liczba naturalna & + \\
			\hline
			name &  Nazwa paszy & max. znaków 50 & + \\
			\hline
			producent &  & max. znaków 50 & - \\
			\hline
			capacity & Ilość paszy w jednym worku & Liczba naturalna & - \\
			\hline
		\end{tabular}
		\caption{Wykaz atrybutów encji typu Meal }
	\end{table}
	Klucze kandydujące: forageID \\
	Klucz główny: forageID \\
	Charakter encji: encja silna \\
	
	
	\item \textbf{Horse}
	
	\textit{Semantyka encji} - Encja zawierająca informacje o koniach.
	
	\begin{table}[h!]
		\centering
		\begin{tabular}{|l|l|l|c|}
			\hline
			Nazwa atrybutu & Opis atrybutu & Typ & OBL(+) \\
			& & &  OPC(-) \\
			\hline
			horseID & Numer identyfikujący konia & Liczba naturalna & + \\
			\hline
			name &  Imie konia & max. znaków 50 & + \\
			\hline
			mother &  Imie klaczy & max. znaków 50 & + \\
			\hline
			father &  Imie ogiera & Max. znaków 50 & - \\
			\hline
			birthday & Data urodzenia konia & Datetime & - \\
			\hline
			race &  Rasa konia & Max. znaków 50 & - \\
			\hline
			breeder &  Hodowca koni & Max. znaków 50 & - \\
			\hline
			passport & Paszport konia & Max. znaków 50 & - \\
			\hline
			photo & Zdjęcie konia & Typ znakowy & - \\
			\hline
		\end{tabular}
		\caption{Wykaz atrybutów encji typu Horse }
	\end{table}
	Klucze kandydujące: horseID \\
	Klucz główny: horseID \\
	Charakter encji: encja silna \\
	
\end{enumerate}
\begin{enumerate}[start=10,label={\bfseries ENC$\backslash$0\arabic*}]
	
	\item \textbf{HorseGender}
	
	\textit{Semantyka encji} - Encja słownikowa zawierająca płeć koni.
	
	\begin{table}[h!]
		\centering
		\begin{tabular}{|l|l|l|c|}
			\hline
			Nazwa atrybutu & Opis atrybutu & Typ & OBL(+) \\
			& & &  OPC(-) \\
			\hline
			genderID & Numer identyfikujący płeć konia & Liczba naturalna & + \\
			\hline
			gender &  Nazwa płci konia & max. znaków 50 & + \\
			\hline
		\end{tabular}
		\caption{Wykaz atrybutów encji typu HorseGender }
	\end{table}
	Klucze kandydujące: genderID \\
	Klucz główny: genderID \\
	Charakter encji: encja silna \\
	
	
	\item \textbf{HorseStatus}
	
	\textit{Semantyka encji} - Encja słownikowa zawierająca statusy koni.
	
	\begin{table}[h!]
		\centering
		\begin{tabular}{|l|l|l|c|}
			\hline
			Nazwa atrybutu & Opis atrybutu & Typ & OBL(+) \\
			& & &  OPC(-) \\
			\hline
			statusID & Numer identyfikujący status konia & Liczba naturalna & + \\
			\hline
			name &  Nazwa statusu konia & max. znaków 50 & + \\
			\hline
		\end{tabular}
		\caption{Wykaz atrybutów encji typu HorseStatus }
	\end{table}
	Klucze kandydujące: statusID \\
	Klucz główny: statusID \\
	Charakter encji: encja silna \\
	
	
	\item \textbf{Meal} \\
	\textit{Semantyka encji} - Encja słownikowa zawierająca nazwy posiłków.
	
	\begin{table}[h!]
		\centering
		\begin{tabular}{|l|l|l|c|}
			\hline
			Nazwa atrybutu & Opis atrybutu & Typ & OBL(+) \\
			& & &  OPC(-) \\
			\hline
			mealID & Numer identyfikujący posiłek & Liczba naturalna & + \\
			\hline
			mealName & Nazwa posiłku & max. znaków 50 & + \\
			\hline
		\end{tabular}
		\caption{Wykaz atrybutów encji typu Meal }
	\end{table}
	Klucze kandydujące: mealID \\
	Klucz główny: mealID \\
	Charakter encji: encja silna \\

\item \textbf{NutritionPlan}

\textit{Semantyka encji} - Encja zawierająca informacje o planie żywienia koni.

\begin{table}[h!]
	\centering
	\begin{tabular}{|l|l|l|c|}
		\hline
		Nazwa atrybutu & Opis atrybutu & Typ & OBL(+) \\
		& & &  OPC(-) \\
		\hline
		nutritionPlanID & Numer identyfikujący plan żywienia & Liczba naturalna & + \\
		\hline
		title &  Tytuł planu żywienia & max. znaków 50 & + \\
		\hline
		desctription &  Ilość jedzenia w porcji & Liczba zmiennoprzecinkowa & - \\
		\hline
		icon &  Ikona dołączona do planu żywienia & Liczba naturalna & + \\
		\hline
	\end{tabular}
	\caption{Wykaz atrybutów encji typu Feed }
\end{table}
Klucze kandydujące: nutritionPlanID \\
Klucz główny: nutritionPlanID \\
Charakter encji: encja silna \\
	\item \textbf{PeopleDetails}\\ \\
	\textit{Semantyka encji} - encja zawiera szczegółowe dane użytkowników (członków klubu, trenerów i zarządu klubu) jak i lekarzy oraz kowali.
	
	\begin{table}[h!]
		\centering
		\begin{tabular}{|l|l|l|c|}
			\hline
			Nazwa atrybutu & Opis atrybutu & Typ & OBL(+) \\
			 & & &  OPC(-) \\
			\hline
			detailsID & Numer identyfikujący dane użytkowników & Liczba naturalna & + \\
			\hline
			name & Imie & max. znaków 50 & - \\
			\hline
			surname & Nazwisko & max. znaków 50 & + \\
			\hline
			phonNumber & Numer telefonu & max. znaków 50 & - \\
			\hline
			email & Adres e-mailowy & max. znaków 50 & - \\
			\hline
			city & Miasto zamieszkania & max. znaków 50 & - \\
			\hline
			street & Ulica zamieszkania & max. znaków 50 & - \\
			\hline
			number & Numer domu zamieszkania & max. znaków 50 & - \\
			\hline
		\end{tabular}
		\caption{Wykaz atrybutów encji typu PeopleDetails }
	\end{table}
Klucze kandydujące: detailsID \\
Klucz główny: detailsID \\
Charakter encji: encja słaba \\

\item \textbf{TakePart}\\ \\
\textit{Semantyka encji} - encja zawierająca dane o zawodach.

\begin{table}[h!]
	\centering
	\begin{tabular}{|l|l|l|c|}
		\hline
		Nazwa atrybutu & Opis atrybutu & Typ & OBL(+) \\
		& & &  OPC(-) \\
		\hline
		takePartID & Numer identyfikujący udział w zawodach & Liczba naturalna & + \\
		\hline
		level & Poziom konkursu & max. znaków 50 & + \\
		\hline
		result & Wynik zawodów & Typ znakowy & + \\
		\hline
		place & Zajęte miejsce &  Liczba naturalna & + \\
		\hline
	\end{tabular}
	\caption{Wykaz atrybutów encji typu Competition }
\end{table}
Klucze kandydujące: competitionID \\
Klucz główny: competitionID \\
Charakter encji: encja silna \\

\item \textbf{Shared}

\textit{Semantyka encji} - Encja zawierająca wpisy o udostępnianiu koni.

\begin{table}[h!]
	\centering
	\begin{tabular}{|l|l|l|c|}
		\hline
		Nazwa atrybutu & Opis atrybutu & Typ & OBL(+) \\
		& & &  OPC(-) \\
		\hline
		sharedID & Numer identyfikujący status konia & Liczba naturalna & + \\
		\hline
		code &  Kod z kodu QR & max. znaków 50 & + \\
		\hline
		endDate &  Data kończąca udostępnienie & max. znaków 50 & + \\
		\hline
		startDate &  Data udostępnienia & max. znaków 50 & + \\
		\hline
	\end{tabular}
	\caption{Wykaz atrybutów encji typu Shared }
\end{table}
Klucze kandydujące: sharedID \\
Klucz główny: sharedID \\
Charakter encji: encja silna \\


\item \textbf{UnitOfMeasure}

\textit{Semantyka encji} - Encja słownikowa zawierająca nazwy jednostek miary.

\begin{table}[h!]
	\centering
	\begin{tabular}{|l|l|l|c|}
		\hline
		Nazwa atrybutu & Opis atrybutu & Typ & OBL(+) \\
		& & &  OPC(-) \\
		\hline
		unitID & Numer identyfikujący jednostkę miary & Liczba naturalna & + \\
		\hline
		unitName & Nazwa jednostek miary & max. znaków 50 & + \\
		\hline
	\end{tabular}
	\caption{Wykaz atrybutów encji typu Meal }
\end{table}
Klucze kandydujące: unitID \\
Klucz główny: unitID \\
Charakter encji: encja silna \\


	\item \textbf{UserAcount}\\ \\
\textit{Semantyka encji} - encja zawiera dane użytkownika (członków klubu, trenerów i zarządu klubu).
\begin{table}[h!]
	\centering
	\begin{tabular}{|l|l|l|c|}
		\hline
		Nazwa atrybutu & Opis atrybutu & Typ & OBL(+) \\
		& & &  OPC(-) \\
		\hline
		userID & Numer identyfikujący użytkownika & Liczba naturalna & + \\
		\hline
		acountLogin & Login użytkownika & max. znaków 50 & + \\
		\hline
		hash & ??? & max. znaków 50 & + \\
		\hline
		salt & ??? & max. znaków 50 & + \\
		\hline
		createdDateTime & Data utworzenia konta & Data & + \\
		\hline
	\end{tabular}
	\caption{Wykaz atrybutów encji typu UserAcount }
\end{table}
Klucze kandydujące: userID \\
Klucz główny: userID \\
Charakter encji: encja słaba \\

\item \textbf{UserType} \\ \\
\textit{Semantyka encji} - encja zawiera typy użytkowników: zwykły użytkownik (standard), trener (trainer), zarząd klubu (admin).
\begin{table}[h!]
	\centering
	\begin{tabular}{|l|l|l|c|}			
		\hline
		Nazwa atrybutu & Opis atrybutu & Typ & OBL(+) \\
		& & &  OPC(-) \\
		\hline
		userTypeID & Numer identyfikujący typ użytkownika & Liczba naturalna & + \\
		\hline
		typeName & Nazwa typu użytkownika & max. znaków 50 & + \\
		\hline
	\end{tabular}
	\caption{Wykaz atrybutów encji typu UserType }
\end{table}
Klucze kandydujące: userTypeID \\
Klucz główny: userTypeID \\
Charakter encji: encja silna \\

\item \textbf{Visit}\\ \\
\textit{Semantyka encji} - encja zawierająca dane o wizytach. 

\begin{table}[h!]
	\centering
	\begin{tabular}{|l|l|l|c|}
		\hline
		Nazwa atrybutu & Opis atrybutu & Typ & OBL(+) \\
		& & &  OPC(-) \\
		\hline
		careID & Numer identyfikujący wizyte & Liczba naturalna & + \\
		\hline
		cost & Cena wizyty & Liczba rzeczywista dodatnia & + \\
		\hline
		summary & Opis podsumowujący wizytę & Typ znakowy & - \\
		\hline
		artefactImage & Zdjęcie z wizyty & Typ znakowy & - \\
		\hline
		visitDate & Data wizyty & Data & + \\
		\hline
	\end{tabular}
	\caption{Wykaz atrybutów encji typu Visit }
\end{table}
Klucze kandydujące: careID \\
Klucz główny: careID \\
Charakter encji: encja silna \\


\end{enumerate}
Po zaprojektowaniu encji możemy zapisać predykatowe defincje typów encji.
\begin{enumerate}[start=1,label={\bfseries ENC$\backslash$00\arabic*}]
	\item 
\end{enumerate}
\section{Model logiczny}
\section{Model fizyczny}
\chapter{Projekt systemu}
\section{Model projektowanego systemu}
\subsubsection{Diagramy stanów}
\subsubsection{Diagramy aktywności}
\subsubsection{Diagram klas}
\subsubsection{Architektura aplikacji}
Jaka baza jakie połączenie itp.
\subsubsection{Wykorzystane wzorce projektowe}
\subsubsection{Model architektoniczny MVVM}
\section{Wybrane aspekty implementacyjne}
jeden viewmodel obsługuje dwa widoki (dodawanie aktywności i szczegóły aktywności) \\
kontrolki\\
\chapter{Testy aplikacji}
\section{Unit testy}
\section{Test case}
\section{Baza błędów}
\chapter{Dokumentacja użytkownika}
\section{Aplikacja desktopowa}
\section{Aplikacja mobilna}
\chapter{Podsumowanie}
\begin{thebibliography}{9}
	
	\bibitem{bazydanych}
	Hanna Mazur, Zygmunt Mazur,
	\emph{Projektowanie relacyjnych baz danych}.
	Oficyna Wydawnicza Politechniki Wrocławskiej, Wrocław 2004.
	
\end{thebibliography}
\listoffigures
\chapter{Spis listingów}
\listoftables
\chapter{Opis zawartości APD}


\end{document}