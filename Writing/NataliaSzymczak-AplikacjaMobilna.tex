\documentclass[12pt,twoside]{report}
\usepackage[margin=2.5cm]{geometry}
\usepackage[utf8]{inputenc}
\usepackage[T1]{fontenc}
\usepackage{polski}
\usepackage{indentfirst}


\begin{document}
	\begin{titlepage}
		
		\begin{center}
			\begin{figure}[t]
				\centering
			%	\includegraphics[width=4.5cm,height=4.5cm]{LogoUO.jpg}
			\end{figure}
		\end{center}
		
		\begin{center}
			{\LARGE  \bf \textsc{UNIWERSYTET OPOLSKI}}
		\end{center}
		\vspace{0.2cm}
		\begin{center}
			{\large \textsc{Wydział Matematyki, Fizyki i Informatyki}}
		\end{center}
		%\vspace{0.2cm}
		\begin{center}
			{\Large \textsc{Instytut Informatyki}}
		\end{center}
		\vspace{0.5cm}
		\begin{center}
			\large    \textsc{Praca iżynierska}
		\end{center}
		\vspace{0.4cm}
		\begin{center}
			\large \textbf{Natalia Szymczak}
		\end{center}
		
		\vspace{0.4cm}
		\begin{center}
			\Large     \textbf{Aplikacja mobilna dla posiadaczy koni}
		\end{center}
		\vspace{0.1cm}
		
		\begin{center}
			\large     \textsc{Mobile aplication for horse owner}
		\end{center}
		\vspace{1.3cm}
		
		\begin{flushright}
			{\large Praca wykonana pod kierunkiem\bigskip
				
				{\bf }} 
			dr Jacka Iwańskiego
		\end{flushright}
		\vspace{0.7cm}
		\begin{center}
			{\large OPOLE 2022}
		\end{center}
	\end{titlepage}

	\thispagestyle{empty}
	\mbox{}
	
	\begin{quote}{\small 
			\noindent
			
			\bigskip
			\noindent
			\textbf{Streszczenie:} 
			
			
			\noindent
			\newline
			\textbf{}
			\vspace{5pt}
			
			\noindent
			\newline
			\textbf{Abstract:} 
			\vspace{5pt}
			
			\vspace{5pt}
			\noindent
			\newline
			\textbf{Keywords:} 
			\vspace{5pt}
			\bigskip
			
			\noindent 
			\textbf{Klasyfikacja tematyczna wg  MSC 2020:}}
	\end{quote}

	\mbox{}
	
	\pagestyle{plain}
	\tableofcontents
	\thispagestyle{empty}
	
	
	\newpage
	\setcounter{page}{1}
	\newpage
\chapter*{Wstęp}
\chapter{Wprowadzenie}
\section{Geneza powstania pracy}
\section{Przegląd istniejących rozwiązań}
Na rynku dostępne jest wiele aplikacji dedykowanych właścicielom koni. Wiele aplikacji poświęconych jest monitoriwaniu treningów. Jest to funkcjonalność, której nie ma w stworzonej aplikacji, więc aplikacje tego typu nie będą analizowane. Przygotowane aplikacje zawierają podobne funkcjonalności do aplikacji "HorseApp". Oprócz opisu działania aplikacji oraz analizy podobieństw, przeanalizowane zostaną mocne i słabe strony każdej z aplikacji.

FEI HorseApp
Lician Horse
Ridely
The equestrian
\section{Cel pracy}

\chapter{Projekt aplikacji}
\section{Wymagania funkcjonalne i niefunkcjonalne}
Funkcjonalności aplikacji mobilnej oraz desktopowej nie są takie same mimo iż są podłączone do jednej bazy, więc czerpią z tego samego źródła informacji. Pomimo różnic niektóre funkcjonalności pokryją się w obu tych produktach.
\\ \\
Wymagania funkcjonalne które muszą spełniać obie aplikacje:
\begin{itemize}
	\item logowanie do aplikacji,
	\item resetowanie hasła przez e-mail,
\end{itemize}
Aplikacja mobilna będzie służyć użytkownikom głównie do zapisu aktualnych wydarzeń z życia stajni. Jej głównym celem jest szybkie zapisanie informacji o aktywnościach koni i ich wizytach u lekarzy bądź kowali. Można w niej także szybko sprawdzić przygotowany plan żywienia, oraz daty zbliżających się zawodów.\\ \\
Wymagania funkcjonalne dla aplikacji mobilnej:
\begin{itemize}
	\item zarządzanie aktywnościami (dodawanie, edytowanie, usuwanie)
	\item wyświetlanie dodanych aktywności oraz ich szczegółowych informacji,
	\item planowanie wizyt,
	\item zarządzanie wizytami (dodawanie, edytowanie, usuwanie)
	\item wyświetlanie szczegółów wizyt,
	\item zapisywanie zdjęć z wizyt lekarzy,
	\item przypomnienia o wizytach,
	\item sprawdzenie/wybór planu żywienia,
	\item sprawdzenie dat zawodów,
	\item potwierdzanie udziału w zawodach,
	\item udostępnianie swojego konia innym użytkownikom aplikacji
\end{itemize}
Aplikacja desktop-owa przeznaczona jest zarówno dla użytkowników posiadających swoje konie jak i dla osób zarządzających klubem jeździeckim. W aplikacji desktop-owej posiadacze koni będą mogli obejrzeć zgromadzone informacje w przystępniejszej formie na dużym ekranie, stworzyć plan żywienia swojego konia, jak także przeanalizować statystki swoich koni. Osoby zarządzające klubem będą miały możliwość dodawania nowych użytkowników i koni jak także sprawdzania statystyk wszystkich koni klubowych.
\\ \\
Wymagania funkcjonalne dla aplikacji desktopowej:
\begin{itemize}
	\item zarządzanie kontami użytkowników,
	\item tworzenie planów żywienia,
	\item zarządzanie końmi,
	\item tworzenie statystyk aktywności,
	\item przeglądanie historii wizyt,
	\item planowanie zawodów
\end{itemize}
Jakiś krótki tekst
\\ \\
Wymagania niefunkcjonalne:
\newpage
\section{Baza danych}
\subsection{Model konceptualny}
\subsection{Model logiczny}
\subsection{Model fizyczny}
\section{Przypadki użycia}
\subsection{Diagram przypadków użycia}
\subsection{Scenariusze przypadków użycia}
\section{Diagramy stanów}
\section{Diagramy aktywności}
\section{Diagram klas}
\section{Architektura aplikacji}
Jaka baza jakie połączenie itp.
\chapter{Implementacja}

\section{Środowisko programistyczne}
\section{Technologie użyte w pracy}
\section{Wykorzystane wzorce projektowe}
\section{Model architektoniczny MVVM}
\section{Interfejs użytkownika}
\section{Wybrane aspekty implementacyjne}
jeden viewmodel obsługuje dwa widoki (dodawanie aktywności i szczegóły aktywności) \\
kontrolki\\
\chapter{Testy aplikacji}
\section{Unit testy}
\section{Test case}
\section{Baza błędów}
\chapter{Podsumowanie}
\chapter{Bibliografia}
\chapter{Spis rysunków}
\chapter{Spis listingów}
\chapter{Spis tabel}
\chapter{Opis zawartości APD}


\end{document}